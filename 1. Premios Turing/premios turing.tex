\documentclass[a4paper, 11pt]{article}
\usepackage[spanish]{babel}
\usepackage[left=3.00cm, right=2.5cm, top=3.50cm, headheight=14pt]{geometry}


\begin{document}
\noindent
\normalsize IPN \hfill \textbar \hfill ESCUELA SUPERIOR DE CÓMPUTO \hfill \textbar \hfill TEORÍA COMPUTACIONAL \\
Por Rubio Haro, Rodrigo R.\hfill 10 de Agosto de 2019 \\
 
\begin{center}
    \large\textbf{Premio Turing \textbar \hspace{.1mm} A.M. Turing Award} \\
\end{center}


\section*{Resumen}
%https://www.turing.org.uk/publications/dnb.html
El premio Turing, a veces referido como el ``Nobel de la Computación", recibe el nombre en honor del Matemático y Científico Británico, Alan Mathison Turing (1912-1954). Avances fundamentales en arquitectura computacional, complejos algoritmos, inteligencia artificial y formalización de la Computación son tan solo unos de sus logros y áreas de estudio. Reconocido por su papel en la segunda guerra mundial en el área de criptografía. Con dudosas causas de muerte, el veredicto fue suicidio, envenenamiento por cianuro, encontrado en junio de 1945. Después de ser llevado a juicio por relaciones homosexuales en la ciudad de Manchester.
%https://abc.xyz/
Patrocinado por Alphabet Inc. (Antes Google Inc.) es premio más prestigioso que la Asociación de Maquinaria Computacional (ACM, por sus siglas en inglés) otorga a las contribuciones más trascendentales e importantes en materia de Computación.


\section*{Lista de premios Turing}

\section*{1966 - Perlis, Alan J.}
\noindent Pennsylvania, 1922. Nace Alan en Pittsburgh, Estados Unidos.

\noindent Ganador del premio por su influencia en el área de técnicas de programación avanzada y construcción de Compiladores. Véase ``The Synthesis of Algorithmic Systems". 

\noindent Áreas de estudio: Compiladores y programación

\noindent Muere en Febrero de 1990, en connecticut, Estados Unidos.
\newline

\section*{1967 - Maurice V. Wilkes.}
\noindent Dudley, 1913. Maurice Vicent Wilkes. en Inglaterra, Reino Unido. 

\noindent Ganador del premio por diseñar la EDSAC, la primera computadora con un programa almacenado y por su coautoría en ``Preparación de programas para computadoras digitales electrónicas". Véase ``The Computers Then and Now ".

\noindent Áreas de estudio: Hardware y Arquitectura Computacional

\noindent Muere en Noviembre de 2010, en Cambriedge, Reino Unido.
\newline

\section*{1968 - Richard W. Hamming.}
\noindent Chicago, 1915. Nace Richard en Illinois, Estados Unidos.

\noindent Ganador del premio por su trabajo en métodos numéricos, códigos de codificación automática, detección y corrección de errores. 

\noindent Áreas de estudio: Códigos de Corrección de Errores y Métodos Numéricos.

\noindent Muere en Enero de 1998, en California, Estados Unidos.
\newline





\end{document}
