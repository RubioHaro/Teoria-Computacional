\documentclass[notitlepage,letterpaper, 11pt]{article}
\usepackage[spanish]{babel}
\usepackage[left=3.00cm, right=2.5cm, top=3.50cm, headheight=14pt]{geometry}
\usepackage{pdfpages}


% Datos de la portada
\title{Premio Turing \textbar \hspace{.1mm} A.M. Turing Award}
\author{Rodrigo R. Rubio Haro}
\date{\today} %

\begin{document}
\noindent
\large\textbf{TAREA 3} \\
\textbf{Brief Notes and History Computing in Mexico
during 50 years} \\
\normalsize IPN \textbar ESCUELA SUPERIOR DE COMPUTO \textbar TEORÍA COMPUTACIONAL \\
Alumno: Rubio Haro, Rodrigo R.\hfill Fecha de Entrega: 13/08/2019 \\
 

\section*{Harold V. McIntosh}

\noindent Nació en Colorado en 1929. Asistió a Brighton High School en Brighton, cerca de Denver, Estados Unidos. 
En 1949 recibió una licenciatura en física del Colegio Agrícola y Mecánico de Colorado, y en 1952 recibió una maestría en matemáticas de la Universidad de Cornell. Realizó más estudios de posgrado en Cornell y Brandeis, pero se detuvo antes de recibir un Ph.D. para tomar un trabajo en el campo de pruebas de Aberdeen. Dos años después, se mudó a RIAS (Instituto de Investigación de Estudios Avanzados), una división de la Compañía Glenn L. Martin. Alrededor de 1962 aceptó un puesto en el departamento de Física y Astronomía y el Proyecto de Teoría Cuántica en la Universidad de Florida. Después de dos años en la Universidad de Florida, McIntosh aceptó una oferta en el CENAC (Centro Nacional de Calculo, Instituto Politécnico Nacional) en México. En los años siguientes, McIntosh trabajó en varios puestos dentro del Instituto Politécnico Nacional, el Instituto Nacional de Energía Nuclear y, desde 1975, en la Universidad Autónoma de Puebla.

\noindent Tuvo una excelente vocación para la enseñanza y la educación, lo que significó un gran reconocimiento por parte de sus alumnos y los centros educativos. Por esta razón, recibió un fuerte apoyo de Roberto Mendiola, director de la Escuela de Física y Matemáticas (ESFM), donde McIntosh fue profesor de 1966 a 1975.

\noindent La investigación informática en el INEN representó el valor de la enseñanza en el ESFM, como uno de los mayores legados de McIntosh en México.
En 1976, Luis Rivera Terrazas, jefe de la UAP (Universidad Autónoma de Puebla), contrató a McIntosh y a un grupo de 12 investigadores del INEN para consolidar la licenciatura.

\newpage
\section*{50 Años de computación en México.}

\noindent REC (compilador de expresiones regulares) era un lenguaje de programación relevante que paso a la historia en la ESFM.
\noindent Con un deseo pedagógico, se utilizó como modelo de lenguaje de una máquina de estados finitos para procesar cuatro símbolos de control
 basados en expresiones regulares.

\noindent La primera computadora electrónica en México fue una IBM 650 instalada en la Universidad Nacional Autónoma de México (UNAM), en junio de 1958. El equipo responsable de este proyecto fue un grupo seleccionado de investigadores en Ingeniería, Física y Matemáticas que trabajaban en funciones clave en la UNAM con  Sergio Beltran como líder del grupo de investigación 

\noindent Beltrán mostró su creatividad, entusiasmo y dedicación organizando el centro de computo con estudiantes de ingeniería y física, capacitándolos con el apoyo de IBM y organizando un coloquio anual para investigadores y estudiantes con algunos de los mejores investigadores de la época. De 1959 a 1962, Beltrán organizó cuatro coloquios internacionales de informática, entre los principales oradores invitados que encontramos, Alan Perlis, McCarthy, Minsky, McIntosh y Niklaus Wirth.

\noindent Durante los años 60, otras universidades como el IPN y la institución privada, El Instituto Tecnológico de Monterrey (ITESM), instalaron sus propios centros de computación. Los programas de capacitación de los centros de computación académicos, sumados a los de la industria informática, permitieron a las primeras instituciones gubernamentales y algunas pocas empresas privadas instalar sus propios centros informáticos.

\noindent A principios de la década de 1980 había dos grupos principales de investigación con alrededor de 22 investigadores cada uno, uno en la UNAM y el otro en la UAP. 

\noindent Para el año de 1984, los dos principales grupos de investigación en informática se redujeron de 22 a 4 investigadores cada uno como resultado del impacto de la crisis económica en la educación superior y los sistemas científicos.

\noindent A pesar de la falta de inversión en la investigación científica en el país, surgieron diversas organizaciones que solamente con la tenacidad, amor al arte y dedicación pudieron lograr ser reconocidas no solo en México, sino internacionalmente.
En 1986 se funda la Asociación Mexicana de Inteligencia Artificial (SMIA), 1995 se creó la Asociación Mexicana de Ciencias de la Computación. Desde entonces, otras áreas han lanzado sus propias asociaciones: robótica, interacción hombre-máquina, procesamiento del lenguaje natural, etc. Este esfuerzo de agrupación ha llevado a la creación en 2015 de la Academia Mexicana de Computación (AMEXCOMP), que es oficialmente reconocida y respaldada por Consejo Nacional de Ciencia y Tecnología (CONACyT).

\noindent Cabe resaltar, también, que en 2018, se se realizó la construcción de la primera máquina Turing en México, auxiliada con legos, y la primera maquina robótica de Turing en el mundo. El trabajo de 2 años, fue producto de la colaboración de la Escuela Superior de Computo del IPN y la UWE de Japón.
La máquina robótica de Turing se presentó este año en la "Conferencia Internacional sobre Vida Artificial y Robótica 2019" (ICAROB2019), Oita en Japón. Por cierto, el fundador de los robots Cubelets, Eric Schweikardt, comenta sobre esta máquina en el foro de Robótica Modular con una publicación titulada "¿Se puede hacer una computadora con Cubelets?".

\newpage
\section*{Adolfo Guzmán Arenas}
Es Ingeniero en Comunicaciones y Electrónica de la Escuela Superior de Ingeniería y Mecánica y Eléctrica del Instituto Politécnico Nacional (IPN). Obtuvo su Maestría y su Doctorado en Ciencias de la Computación en el Instituto Tecnológico de Massachusetts (MIT), en Cambridge, Massachusetts, EE.UU. Fue profesor del Departamento de Ingeniería Eléctrica del MIT; del Departamento de Inteligencia Mecánica de la Universidad de Edimburgo; del Centro de Investigación y Estudios Avanzados del IPN, donde fundó la Maestría y Doctorado en Computación; del Instituto de Investigación en Matemáticas Aplicadas y Sistemas, de la UNAM, donde fue Jefe del Departamento de Computación; y de la Unidad Interdisciplinaria (UPIICSA) del IPN. Fue Director del Centro Científico IBM para América Latina, IBM de México, S.A. Ha sido Investigador Senior de la empresa MicroElectronics and Computer Corporation (MCC); Vicepresidente de Ingeniería en International Software Systems, y fundador y Presidente de SoftwarePro International, empresa en Austin, Texas, dedicada al desarrollo de paquetes comerciales y herramientas de Ingeniería de Software. En 1994 recibió el Premio Nacional de Informática, que otorga la Academia Mexicana de Informática. Recibió en 1996 de manos del Presidente Zedillo el Premio Nacional de Ciencias y Artes. Y de sus mismas manos, en 1997, la Presea “Lázaro Cárdenas”. Fundó en 1996 el Centro de Investigación en Computación (CIC) del IPN y lo dirigió hasta 2002. Adolfo es miembro de la Academia de Ingeniería, la Academia Mexicana de Ciencias y el Consejo Consultivo de Ciencias. Es Doctor Honoris Causa del Instituto Nacional de Astrofísica, Óptica y Electrónica. Es Fellow of the Association for Computing Machinery (ACM), y Fellow of the Institute of Electrical and Electronic Engineers (IEEE). En el CIC trabaja en el uso de Inteligencia Artificial en análisis de textos (y representación del conocimiento), procesamiento semántico y aplicaciones de sistemas de información

\begin{thebibliography}{*}
    \bibitem{1} Brief Notes and History Computing in Mexico during 50 years
    (Genaro J. Martinez, Juan C. Seck-Tuoh-Mora, Sergio V. Chapa-Vergara, Christian Lemaitre). Obtenido Agosto 13, 2019, de https://arxiv.org/abs/1905.07527
    
    \bibitem{TWO}  Adolfo Guzman Arenas. Curriculum Vitae
    (CIC). Obtenido Agosto 13, 2019, de http://www.cic.ipn.mx/aguzman/vitae.html
    
\end{thebibliography}
\end{document}
