\documentclass[letterpaper, 11pt]{article}
\usepackage[spanish]{babel}
\usepackage[left=3.00cm, right=2.5cm, top=3.50cm, headheight=14pt]{geometry}


\begin{document}
%Construcción del encabezado, asegurate de cambiar tus datos !!!!
\noindent
\large\textbf{TAREA 2} \\
\textbf{BIOGRAFIAS} \\
\normalsize IPN \textbar ESCUELA SUPERIOR DE COMPUTO \textbar TEORÍA COMPUTACIONAL \\
Alumno: Rubio Haro, Rodrigo R.\hfill Fecha de Entrega: 12/08/2019 \\
 

\section*{Alan Mathison Turing}
Alan Mathison Turing, nació en la ciudad de Londres, el 23 de junio en 1912. A muy corta edad demuestra sus dotes naturales para las matemáticas y una inteligencia que asombra a los adultos que lo rodean. Estudió en Instituciones como la Universidad de Manchester y de Cambridge.

\noindent En 1937 publicó un célebre artículo el de la ``máquina de Turing" que operaba basándose en una serie de instrucciones lógicas, sentando así las bases del concepto moderno de algoritmo. Turing describió en términos matemáticos precisos cómo un sistema automático con reglas extremadamente simples podía efectuar toda clase de operaciones matemáticas expresadas en un lenguaje formal determinado. La máquina de Turing era tanto un ejemplo de su teoría de computación como una prueba de que un cierto tipo de máquina computadora podía ser construida.

\noindent La Segunda Guerra Mundial ofreció un insospechado marco de aplicación práctica de sus teorías, al surgir la necesidad de descifrar los mensajes codificados que la Marina alemana empleaba para enviar instrucciones a los submarinos que hostigaban los convoyes de ayuda material enviados desde Estados Unidos; Turing, al mando de una división de la Inteligencia británica, diseñó tanto los procesos como las máquinas que, capaces de efectuar cálculos combinatorios mucho más rápidamente que cualquier ser humano, fueron decisivos en la ruptura final del código, algo que le ha hecho ganar popularidad a lo largo de los años.

\noindent Alan Turing definió además un método teórico para decidir si una máquina era capaz de pensar como un hombre (el famoso test de Turing) y realizó contribuciones a otras ramas de la matemática aplicada, como la aplicación de métodos analíticos y mecánicos al problema biológico de la morfogénesis. Trágicamente, su orientación sexual fue motivo constante de fuertes presiones sociales y familiares, hasta el punto de ser llevado a juicio por haber mantenido relaciones homosexuales en la ciudad de Manchester. Con dudosas causas de muerte, el veredicto fue suicidio, envenenamiento por cianuro, encontrado en su cada en junio de 1945.


\newpage
\section*{Emil Leon Post}

Nació en el año 1897 en Agustow, Polonia. Se trasladó a Estados Unidos a los siete años con su familia en 1904 desde Augustów, Polonia. El primer tema por el que se sintió atraido fue la astronomía. Asistió a la Universidad de la ciudad de Nueva York, recibiendo su B.A. en 1917. Aunque el escribió su primer artículo en la Universidad, no fue publicado hasta unos años después. En 1917 empezó a investigar en la Universidad de Columbia, recibiendo su A.M en 1918 y su Ph D. en 1920. En su tesis doctoral probó la consistencia del cálculo proposicional. Su trabajo marcó el principio de la teoría demostracional.

\noindent Después de recibir su doctorado, Post fue a Princeton un año. Regresó a Columbia y, poco después de esto, tuvo el primer ataque de una enfermedad que se repitió a lo largo de su carrera y limitó que pudiera haber triunfado. En 1924 Post fue a Cornell, pero otra vez cayó enfermo. Continuó su trabajo como profesir de instituto en Nueva York en 1927. Entonces en 1932 fue destinado a la Universidad de la ciudad. Lo dejó una corta temporada pero regresó tres años más tarde y pasó el resto de su vida allí.

\noindent Post introdujo el concepto de completitud y consistencia sobre el método de las tablas de verdad. Atribuyó ese método a CJ Keyser antes que a Charles Pierce y E. Schöder como había sido hecho previamente. En la década de los 20 Post obtuvo resultados similares a Gödel, Church y Turing, pero no los publicó. También hizo un estudio matemático de la lógica tri-evaluada de Lukasiewicz. Aproximadamente sobre este periodo escribió en su diario: "Yo estudio Matemáticas como producto de la mente humana, no como categórica."

\noindent En 1936 propuso lo hoy conocido como Máquina de Post, un tipo de autómata que recoge la noción de programa que estudió von Neumann en 1946. En 1941 escribió:

"...el pensamiento matemático es, y debe ser, esencialmente creativo..." pero dijo que había limitaciones y la lógica simbólica es "...el indiscutible método para revelar y descubrir esas limitaciones."

\noindent Post mostró que el problema para semigrupos era recursivamente irresoluble en 1947, dando la solución a un problema planteado por Thue en 1914. Quine, en una carta escrita en 1954 después de la muerte de Post, decía: "la teoría demostracional moderna y del mismo modo, la teoría de la máquina computacional moderna gira en torno al concepto de la función recursiva. Este importante concepto teórico... fue descubierto independientemente...por cuatro matemáticos, y uno de ellos fue Post". Parte del trabajo de Post sería un instrumento para progresos posteriores en las funciones recursivas.

\noindent Murió en el 1954 en Nueva York.

\newpage
\section*{John von Neumann}

Nace en Budapest en 1903 el Matemático húngaro nacionalizado estadounidense John von Neumann. De familia de banqueros judíos, dio muestras desde niño de unas extraordinarias dotes para las matemáticas.

\noindent Se matricula en la Universidad de Budapest en 1921, donde se doctoró en matemáticas cinco años después, aunque pasó la mayor parte de ese tiempo en otros centros académicos: en la Universidad de Berlín asistió a los cursos de Albert Einstein; estudió también en la Escuela Técnica Superior de Zurich, donde en 1925 se graduó en ingeniería química, y frecuentó asimismo la Universidad de Gotinga.

\noindent Allí conoció al matemático David Hilbert (cuya obra ejerció sobre él considerable influencia) y contribuyó de manera importante al desarrollo de lo que Hilbert llamó la teoría de la demostración; aportó además diversas mejoras a la fundamentación de la teoría de conjuntos elaborada por Ernst Zermelo. En Gotinga asistió también al nacimiento de la teoría cuántica de Werner Heisenberg y se interesó por la aplicación del programa formalista de Hilbert a la formulación matemática de esa nueva rama de la física.

\noindent Ello le llevó a convertirse en el autor de la primera teoría axiomática abstracta de los llamados -precisamente por él- espacios de Hilbert y de sus operadores, que a partir de 1923 habían empezado a demostrar su condición de instrumento matemático por excelencia de la mecánica cuántica. La estructura lógica interna de la mecánica cuántica, en efecto, se puso de manifiesto merced a los trabajos de Von Neumann, quien contribuyó a proporcionarle una base rigurosa para su exposición.

\noindent También es notable su apertura de nuevas vías al desarrollo de la matemática estadística a partir de su estudio de 1928 sobre los juegos de estrategia, posteriormente desarrollado en la famosa obra Theory of games and economic behavior, publicada en 1944 y escrita en colaboración con Oskar Morgenstern.

\noindent En 1943, el ejército estadounidense reclamó su participación en el Proyecto Manhattan para la fabricación de las primeras bombas atómicas; a partir de entonces, Von Neumann colaboró permanentemente con los militares, y sus convicciones anticomunistas propiciaron que interviniera luego activamente en la fabricación de la bomba de hidrógeno y en el desarrollo de los misiles balísticos.

\noindent Entre 1944 y 1946 colaboró en la elaboración de un informe para el ejército sobre las posibilidades que ofrecía el desarrollo de las primeras computadoras electrónicas; de su contribución a dicho desarrollo destaca la concepción de una memoria que actuase secuencialmente y no sólo registrara los datos numéricos de un problema, sino que además almacenase un programa con las instrucciones para la resolución del mismo.

\noindent Se interesó también por la robótica, y en 1952 propuso dos modelos de máquinas autorreproductoras, uno de ellos con una modalidad de reproducción parecida a la de los cristales, mientras que el otro era más próximo a la forma en que se reproducen los animales. En 1955, tras solicitar la excedencia de Princeton, fue nombrado miembro de la Comisión de Energía Atómica del gobierno estadounidense; ese mismo año un cáncer en estado muy avanzado lo apartó de toda actividad hasta su muerte.
\newpage
\section*{Marvin Minsky}


El científico y matemático Marvin Lee Minsky, considerado el padre de la inteligencia artificial, nació el 9 de agosto de 1927,  en Nueva York, Estados Unidos. Minsky ganó en 1969 el AM Turing Award, el más alto honor en ciencias de la computación, por su trabajo pionero en la IA.

\noindent Después de prestar servicio en la Marina de Estados Unidos desde 1944 hasta 1945, Minsky se matriculó en 1946 en la Universidad de Harvard para explorar sus múltiples intereses intelectuales. Después de completar investigaciones en física, neurofisiología y psicología, se graduó con honores en matemáticas en 1950. En 1951 ingresó en la Universidad de Princeton, y en ese mismo año construyó el primer simulador de una red neuronal. 
\noindent En 1954, con un doctorado en matemáticas de Princeton, Minsky regresó a Harvard como miembro de la prestigiosa Society of Fellows. Inventó el microscopio confocal de barrido en 1955.
\noindent En 1957 se trasladó al Instituto de Tecnología de Massachusetts (MIT) para seguir su interés en el uso de las computadoras para modelar y comprender el pensamiento humano. Entre otros interesados en la IA estaban John McCarthy, profesor del MIT de ingeniería eléctrica, que había desarrollado el lenguaje LISP de programación informática y contribuyó al desarrollo de sistemas informáticos de tiempo compartido  (sistemas en los que varios usuarios interactúan con una sola computadora central). 

\noindent En 1959 Minsky y McCarthy cofundaron lo que se convertiría en el Laboratorio de Inteligencia Artificial del MIT. Rápidamente se transformó en uno de los centros de investigación más prominentes y centro de capacitación para el naciente campo de la IA. Minsky permaneció en el MIT por el resto de su carrera, adquiriendo los prestigiosos títulos de Profesor Donner de Ciencias en 1974 y profesor Toshiba de Artes y Ciencias del Laboratorio de Medios del MIT en 1990.

\noindent Minsky definía la AI como "la ciencia de hacer que las máquinas hagan cosas que requieren de inteligencia si son hechas por los hombres." A pesar de algunos éxitos iniciales, los investigadores de Inteligencia Artificial encontraban cada vez más difícil de capturar el mundo exterior con la dura sintaxis de los lenguajes de programación de incluso los ordenadores más potentes. 

\noindent En 1975 Minsky desarrolló el concepto de "frames" para identificar con precisión la información general que debe ser programado en un ordenador antes de considerar direcciones específicas. Por ejemplo, si un sistema tenía que navegar a través de una serie de habitaciones comunicadas por puertas, Minsky sugirió que el marco (frame) tendría que articular la gama asociada de posibilidades para las puertas, en otras palabras, todo el conocimiento de sentido común que un niño debe aplicar cuando se enfrenta a una puerta: que la puerta puede oscilar de cualquier forma en una bisagra, que la puerta puede abrirse y cerrarse, y que el pomo de la puerta puede tener que girarse antes de empujar o tirar para abrir la puerta. Los frames demostraron ser un concepto rico entre los investigadores de la IA, aunque su aplicación a situaciones altamente complejas ha resultado difícil.

\noindent Sobre la base de sus experiencias con los frames y la psicología infantil del desarrollo, Minsky escribió The Society of Mind (1987), en la que presentó su visión de la mente como un compuesto de agentes individuales que realizan funciones básicas, tales como el equilibrio, el movimiento, y la comparación. Sin embargo, los críticos sostienen que la idea de la "sociedad de la mente" es más accesible a los legos y apenas útil para los investigadores de IA.

\noindent Marvin Lee Minsky falleció en Boston el 24 de enero de 2016, tras sufrir un ACV.
\newpage
\section*{Alonzo Church}

Alonzo Church, nació en la ciudad de Washington D.C. Profesor en la Universidad de Princeton y en la de California, se especializó en lógica matemática, metalógica y metamatemática.

\noindent De sus trabajos es notable su concepto de calculabilidad de una función y su demostración de la indecidibilidad de la lógica de primer orden, es decir, del cálculo cuantificacional elemental. Algunas de sus ideas serían ampliadas o complementadas por Alan Turing, pionero en la formulación teórica de la informática. Alonzo Church desarrolló asimismo el cálculo de conversión lambda, que permite efectuar operaciones lógicas con variables generalizadas.

\noindent Destacan, dentro de su abundante producción teórica, sus obras Cálculo de conversión lambda (1941) e Introducción a la lógica matemática (1944). Church difundió sus trabajos a través del Journal of Symbolic Logic, publicación que él mismo dirigía y editaba desde 1936.

\noindent Murió en Hudson, Ohio, a la edad de 92 años y fue enterrado en el cementerio de Princeton.


\begin{thebibliography}{*}
    \bibitem{Tring} 
    Peña, R. M. (2013, 9 marzo). La vida y la obra de Alan Turing. Recuperado 12 agosto, 2019, de 
    {http://blogs.mat.ucm.es/shm/wp-content/uploads/sites/17/2013/06/TuringRP\_SHMat-Ene13.pdf}

    \bibitem{POST} 
    Urquhart, A. (2008). EMIL POST. Recuperado 12 agosto, 2019, de https://sites.ualberta.ca/~francisp/Phil428.526/UrquhartPost.pdf


\end{thebibliography}

\end{document}
