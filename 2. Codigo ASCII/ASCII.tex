\documentclass[letterpaper, 11pt]{article}
\usepackage[spanish]{babel}
\usepackage[left=3.00cm, right=2.5cm, top=3.50cm, headheight=14pt]{geometry}


\begin{document}
%Construcción del encabezado, asegurate de cambiar tus datos !!!!
\noindent
\large\textbf{TAREA 2} \\
\textbf{CÓDIGO ASCII} \\
\normalsize IPN \textbar ESCUELA SUPERIOR DE COMPUTO \textbar TEORÍA COMPUTACIONAL \\
Alumno: Rubio Haro, Rodrigo R.\hfill Fecha de Entrega: 12/08/2019 \\
 

\section*{Resumen}
El Código Americano Estándar para el intercambio de Información (ASCII, por sus siglas en inglés) fue creado en 1963 por el Comité Estadounidense de Estándares o ``ASA", ahora ``Instituto Estadounidense de Estándares Nacionales" (``ANSI" por sus siglas en inglés).

\noindent El objetivo era reordenar y expandir el conjunto de símbolos y caracteres ya utilizados en aquel momento en telegrafía por la compañía Bell. En un primer momento solo incluía letras mayúsculas y números, pero en 1967 se agregaron las letras minúsculas y algunos caracteres de control, formando así lo que se conoce como US-ASCII, es decir los caracteres del 0 al 127. 
Así con este conjunto de solo 128 caracteres fue publicado en 1967 como estándar, conteniendo todos lo necesario para escribir en idioma ingles.

\noindent En 1981, la empresa IBM desarrolló una extensión de 8 bits del código ASCII, llamada ``pagina de código 437", en esta versión se reemplazaron algunos caracteres de control obsoletos, por caracteres gráficos. Además se incorporaron 128 caracteres nuevos, con símbolos, signos, gráficos adicionales y letras latinas, necesarias para la escrituras de textos en otros idiomas, como por ejemplo el español. Así fue como se sumaron los caracteres que van del ASCII 128 al 255.
IBM incluyó soporte a esta página de código en el hardware de su modelo 5150, conocido como ``IBM-PC", considerada la primera computadora personal. El sistema operativo de este modelo, el ``MS-DOS" también utilizaba el código ASCII extendido.

\noindent Casi todos los sistemas informáticos de la actualidad utilizan el código ASCII para representar caracteres, símbolos, signos y textos (222) .

\begin{thebibliography}{*}
    \bibitem{Tring} El codigo ASCII. (n.d.). Obtenido Agosto 12, 2019, de https://elcodigoascii.com.ar/
    \end{thebibliography}

\end{document}
